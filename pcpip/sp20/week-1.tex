\documentclass{article}
\usepackage{pcpip-scribe}

\begin{document}

\lecture{1}{Interactive Proofs, dIP, GNI, AM/MA}{Max Ovsiankin}{2/30/2020}

Welcome to the UGTCS PCPIP reading group!
The prerequisites for this reading group are formally only CS 170, although it would definitely
help if you have some background in complexity theory.
Nevertheless, we will try to introduce the relevant complexity-theoretic notions and definitions
as we go along. 

This reading group will also cover some pretty mathematically difficult ideas, so be prepared!
Confusion is a normal and expected part of the process.

\section{Complexity Theory Background}
Complexity theory as we know it from CS 170 is about measuring the difficulty of problems by
the efficiency of the algorithms that solve them.
As a shorthand, we say efficient algorithms are exactly those that are 
polynomial-time (ignoring details such as leading constants or powers).
Why this `extended Church-Turing thesis' holds is a story for another time,
but the class of problems that have poly-time algorithms are called $\Pclass$:

\begin{definition}
    Let $L \subseteq \{0, 1\}^*$. This language $L \in \Pclass$ if there exists a
    polynomial $p\colon \N \to \N$ and Turing machine $M$ such that $M(x)$ runs in time $p(|x|)$, and 
    $$ x \in L \iff M(x) = 1$$
\end{definition}

Languages that are in $\Pclass$ are not very problematic.
From both a complexity and cryptography perspective, you could say the information in
$\Pclass$ languages can be `known' by any party with access to reasonable computational power.

But now, let's say that in a complexity-theoretic setting, you `know' something I don't.
You want to convince me that this thing you know is true.
We already know from CS 170 a complexity class that encodes this idea:

\begin{definition}
    $L \in \NP$ if there exists a polynomial $q \colon \N \to \N$ and Turing machine $V(x, \pi)$ (for verifier) such that
    $V$ runs in time $\poly(|x|)$, and satisfies the following two properties:
    \begin{itemize}
        \item \textbf{Completeness}: If $x \in L$, then there exists proof string $\pi \in \{0, 1\}^{q(|x|)}$ such that $V(x, \pi) = 1$
        \item \textbf{Soundness}: If $x \notin L$, then for all $\tilde{\pi}$, $V(x, \tilde{\pi}) = 0$.
    \end{itemize}
\end{definition}

This may not be the definition of $\NP$ that you recognize, but we've written it in a suggestive manner because we're about to modify it.
You can think of $\NP$ as encoding the notion of a `proof system', where you give me a short proof,
and I do an efficient check on the proof to see if it's correct.
If something is true, you should be able to convince me it is true with some proof (this is completeness).
If something is false, you shouldn't be able to convince me it is true, no matter what proof you give me (this is soundness). 

We know that $\Pclass \subseteq \NP$. 
It is a fundamental open problem whether $\Pclass \subsetneq \NP$, i.e. if there is truly a gap between things that are
poly-time `knowable' and things that you can convince me are true with a short proof string.
Maybe the fact that it only takes a short proof string to convince me means I should be able to know it by myself.
Regardless, it's interesting to explore what happens when we push this notion of `short proof string'.
A great many rich innovations have been made in the 30 years since this thought.

In real life, we have more available to us than just you giving me a piece of paper and me verifying it.
What happens if we have a conversation, instead?

\section{Deterministic Interactive Proofs}

Imagine connecting two Turing machines such that they share a tape, and can talk to each other.
One will be the prover $P$, and the other will be the verifier $V$.

\begin{definition}
The variable $\langle P, V \rangle(x)$ respresents the following interaction (which we call a \textnormal{$k$-round interaction}):
\begin{quote}
\begin{enumerate}
    \item[(0.)] $V$ and $P$ both have access to input $x$
    \item[1.] \textit{Round 1:} $V$ sends message $a_1$ to $P$, $V \stackrel{a_1}{\longrightarrow} P$
    \item[2.] \textit{Round 2:} $P \stackrel{a_2}{\longrightarrow} V$ 
    \item[3.] \textit{Round 3:} $V \stackrel{a_3}{\longrightarrow} P$\\
    \vdots 
    \item[k.] \textit{Round k:} $P \stackrel{a_k}{\longrightarrow} V$
    \item[(n+1).] set $ \langle P, V \rangle(x) = V(x, a_1, \ldots, a_k)$, i.e. $V$'s view after the interaction finishes
\end{enumerate}
\end{quote}
\end{definition}

Lets define a class $\dIP$, which is exactly $\NP$ extended to allow interaction.
We restrict the verifier to be poly-time just like in the definition of $\NP$.
However, just as there's a `magic' proof string $\pi$, we allow the prover $P$ unlimited computational power.

\begin{definition}
    $L \in \dIP$ if there exists a Turing machine $V(x)$ (for verifier) such that
    $V$ runs in time $\poly(|x|)$ for each round, and satisfies the following properties:
    \begin{itemize}
        \item \textbf{Completeness}: If $x \in L$, then there exists unbounded prover $P$ such that $\langle P, V \rangle(x) = 1$
        \item \textbf{Soundness}: If $x \notin L$, then for all unbounded provers $\tilde{P}$, $\langle \tilde{P}, V \rangle(x) = 0$
        \item \textbf{Complexity}: The number of rounds of $\langle P, V \rangle(x)$ is polynomial in $|x|$
    \end{itemize}
\end{definition}

How is $\dIP$ related to $\NP$? Well, $\NP \subseteq \dIP$, as that is exactly a 1-round interaction from the prover to the verifier.
However, it turns out that allowing interaction has not given us anything more!

\begin{claim} $\dIP = \NP$\end{claim}
\begin{proof}
    $\NP \subseteq \dIP$ as remarked above.

    To show $\dIP \subseteq \NP$, let $L \in \dIP$.
    Consider the poly-time verifier $V'$, which takes a transcript $(a_1, \ldots, a_k)$ and checks that
    $a_1 = V(x)$, $a_3 = V(x, a_1, a_2)$, and so on, until finally it checks that $V(x, a_1, \ldots, a_k) = 1$. 
    We will use this transcript as the $\NP$-certificate.

    \textbf{Completness}: If $x \in L$, then there is some transcript $(a_1, \ldots, a_k)$ produced by $P$ that causes $V$ to accept.
    But then $V'$ accepts this transcript. \\
    \textbf{Soundness}: (we show the contrapositive) If there exists some transcript $(a_1, \ldots, a_k)$ that $V'$ accepts, then there is a prover $P$ that emits this transcript, so that $\langle P, V \rangle(x) = 1$. But then $x \in L$.
\end{proof}

\section{Graph Non-Isomorphism}

Before we relax the notion of IPs and hopefully bring them more power, let's explore an interesting problem we believe to be beyond $\NP$.
Consider two graphs $G_1, G_2$ with $V_1 = V_2 = \{1, \ldots, n \}$.
We say these two graphs are \textit{isomorphic} (written $G_1 \cong G_2$) if there exists some permutation $\pi \colon \{1, \ldots, n \} \to \{1, \ldots, n\}$ such that $G_2 = \pi(G_1)$. 
In essence, the two graphs are isomorphic if their vertices are relabelings of each other, or if they `look the same'.

$\textsf{GI}$ is the language consisting of all pairs $(G_1, G_2)$ such that 
$G_1 \cong G_2$. We can see that $\textsf{GI} \in \NP$, as $\pi$ is our short certificate, and $\pi$ can be easily verified as a permutation that modifies the graphs in the correct way. It is not known if $\textsf{GI} \in \Pclass$.

$\textsf{GNI}$ is the exactly the complement: the language consisting of all pairs $(G_1, G_2)$ such that $G_1 \not\cong G_2$. 
$\textsf{GNI}$ is in $\coNP$, but is not believed to be in $\NP$. 
In fact, it's not immediately clear how one can give a short proof, or even an interaction as a witness to the fact that two particular graphs $G_1, G_2$ are \textit{not} isomorphic. 
However, it turns out that if we allow the poly-time verifier to use random coin flips, there is an interactive protocol for $\textsf{GNI}$!

\section{Interactive Proofs}

We are now ready to define interactive proofs:

\begin{definition}
    $L \in \IP$ if there exists a (probabilistic) polynomial-time $V$, such that:
    \begin{itemize}
        \item \textbf{Completeness}: If $x \in L$, then there exists a $P$ where $\Pr[\langle P, V \rangle(x) = 1] = 1$.
        \item \textbf{Soundness}: If $x \notin L$, then for any $\tilde{P}$, we have $\Pr[\langle P, V \rangle(x) = 1] \leq \frac12$.
        \item \textbf{Complexity}: The number of rounds of $\langle P, V \rangle(x)$ is polynomial in $|x|$.
    \end{itemize}
\end{definition}

In this definition, both probabilities are now taken over the random coin flips of $V$.
Note also that we can make soundness exponentially decay with
sequential repetitions of the protocol. More explicitly, if we repeat the protocol $|x|^c$ times so that the soundness error now becomes negligible in $|x|$, that is still in $\IP$.

As hinted in the last section, there is an interactive protocol to show GNI under this definition:

\begin{theorem}[GMW85]
    $\textnormal{\textsf{GNI}} \in \IP$
\end{theorem}
\begin{proof}
Consider the following protocol:
    \begin{center}
    \begin{tikzpicture}[auto,>=stealth']
        \node[] (verifier) {$V$};
        \node[left of= verifier, node distance=3cm] (prover) {$P$};
        \node[below of=verifier, node distance=5cm] (verifier_ground) {};
        \node[below of=prover, node distance=5cm] (prover_ground) {};
        %
        \draw (prover) -> (prover_ground);
        \draw (verifier) -- (verifier_ground);
        \draw[<-] ($(prover)!0.35!(prover_ground)$) -- node[above,scale=1,midway]{Hey} ($(verifier)!0.35!(verifier_ground)$);
        \draw[->] ($(prover)!0.25!(prover_ground)$) -- node[below,scale=1,midway]{Text} ($(verifier)!0.25!(verifier_ground)$);
    \end{tikzpicture}
\end{center}

\end{proof}

\end{document}
